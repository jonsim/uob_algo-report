\documentclass{acm_proc_article-sp}

\usepackage{graphicx}
\usepackage{amsmath}
\usepackage{float}
\usepackage{natbib}
\usepackage{url}
\usepackage{amssymb}

\begin{document}
\title{Analysis of an Adaption of the Adaptive Aggressive Algorithm: Asking are AA Algorithms Actually All that Accurate in Applicable Auctions Anyway?}
\numberofauthors{3} 
\author{
  \alignauthor
    Rupert Bedford\\
    \email{rb9281@bris.ac.uk}
  \alignauthor
    Max Robinson\\
    \email{mr9388@bris.ac.uk}
  \alignauthor
    Jonathan Simmonds
    \email{js9721@bris.ac.uk}
}
\date{7 December 2012}

\maketitle
\begin{abstract}
Abstract goes here.
\end{abstract}

\pagebreak

\section{Introduction}
Rupert

An automated trader is an algorithm that automatically places trading orders on
a stock market.
The trader receives buy or sell orders from customers with a price and
quantity.
The trader makes money by buying for less than the customers price or selling
for more than the customers price.
Therefore the goal of a trading strategy is maximise these margins and the
trading volume to maximise the amount of profit.

IBM published a paper in 2001 that showed that their MGD trading strategy and
Dave Cliff's ZIP were able to outperform human traders.
Algorithmic strategies are able to react more quickly to changing market
conditions and combine information from a multiple sources.

We have implemented the Adaptive Aggressiveness trading algorithm and compare
its performance with other adaptive algorithms as well as simpler traders.

\begin{itemize}
	\item What is automated trading (history)
	\item What our goal was
\end{itemize}


\section{Environment}
\subsection{BSE}
Jonny
\begin{itemize} \itemsep0pt
	\item Limitations
	\item Restrictions (level of info --- sealed bid double auction, dark pools? (\url{http://en.wikipedia.org/wiki/Dark_liquidity}, \url{http://en.wikipedia.org/wiki/Algorithmic_trading#Strategies_that_only_pertain_to_dark_pools}) etc.)
\end{itemize}

\subsection{Traders}
\subsubsection{ZIP}
Jonny
\begin{itemize} \itemsep0pt
	\item What was significant --- key points
	\item Advantages/disadvantages
\end{itemize}

\subsubsection{GD-Variants}
Rupert
\begin{itemize} \itemsep0pt
	\item Shavers / Sniper / XKCD, etc.
	\item We implemented it but couldn't do a full implementation
\end{itemize}

\subsection{Adaptive Aggressive Traders}
\subsubsection{Short term learning}
Max
\begin{itemize} \itemsep0pt
	\item Graphs --- \tt r \rm vs price equilibrium
\end{itemize}

\subsubsection{Long term learning}
Max
\begin{itemize} \itemsep0pt
	\item $\theta$
\end{itemize}

\subsubsection{Price estimation}
Max
\begin{itemize} \itemsep0pt
	\item Graph of all trades with projected price equilibrium
	\item \textbf{MEGA GRAPH}
\end{itemize}


\section{Calibration}
Group hug
\begin{itemize} \itemsep0pt
	\item $\beta_1$, $\beta_2$, $\gamma$, $\eta$
	\item potential to compare statistically?
\end{itemize}


\section{Results}
Group hug
\begin{itemize} \itemsep0pt
	\item Graph: Average balance over time
	\item Statistical \textbf{anal}ysis --- why you used a certain test\\
Ed's report: ``According to the conducted Wilcoxon-Mann-Whitney two-tailed rank-sum tests, the difference in the observed efficiencies is significant ($U = 2, N_1 = N_2 = 10, p < 0.0003$)."
	\item Experiment with changing scheduler
	\item Other graphs
\end{itemize}

\section{Conclusion}
Group hug
\begin{itemize} \itemsep0pt
	\item Thank you and good night
	\item Hold for applause
\end{itemize}


%\pagebreak
%\bibliographystyle{unsrt}
%\bibliography{References}
\end{document}
