\documentclass{acm_proc_article-sp}

\usepackage{graphicx}
\usepackage{amsmath}
\usepackage{float}
\usepackage{natbib}
\usepackage{url}
\usepackage{amssymb}

\begin{document}
\title{Analysis of an Adaption of the Adaptive Aggressive Algorithm: Asking are AA Algorithms Actually All that Accurate in Applicable Auctions Anyway?}
\numberofauthors{3} 
\author{
  \alignauthor
    Rupert Bedford\\
    \email{rb9281@bristol.ac.uk}
  \alignauthor
    Max Robinson\\
    \email{mr9388@bristol.ac.uk}
  \alignauthor
    Jonathan Simmonds
    \email{js9721@bristol.ac.uk}
}
\date{7 December 2012}

\maketitle
\begin{abstract} \label{sec:abstract}
\begin{verbatim}
                             .::::. 
                           .::::::::. 
                           ::::::::::: 
                           ':::::::::::.. 
                            :::::::::::::::' 
                             ':::::::::::. 
                               .::::::::::::::' 
                             .:::::::::::... 
                            ::::::::::::::'' 
                .:::.       '::::::::'':::: 
              .::::::::.      ':::::'  ':::: 
             .::::':::::::.    :::::    '::::. 
           .:::::' ':::::::::. :::::      ':::. 
         .:::::'     ':::::::::.:::::       '::. 
       .::::''         '::::::::::::::       '::. 
      .::''              '::::::::::::         :::... 
   ..::::                  ':::::::::'        .:' '''' 
..''''':'                    ':::::.' 
\end{verbatim}
\end{abstract}

\pagebreak

\section{Introduction} \label{sec:introduction}
Rupert

An automated trader is an algorithm that automatically places trading orders on
a stock market.
The trader receives buy or sell orders from customers with a price and
quantity.
The trader makes money by buying for less than the customers price or selling
for more than the customers price.
Therefore the goal of a trading strategy is maximise these margins and the
trading volume to maximise the amount of profit.

IBM published a paper in 2001 that showed that their MGD trading strategy and
Dave Cliff's ZIP were able to outperform human traders.
Algorithmic strategies are able to react more quickly to changing market
conditions and combine information from a multiple sources.

We have implemented the Adaptive Aggressiveness trading algorithm and compare
its performance with other adaptive algorithms as well as simpler traders.

\begin{itemize}
	\item What is automated trading (history)
	\item What our goal was
\end{itemize}


\section{Environment} \label{sec:environment}
\subsection{BSE} \label{sec:BSE}
For this assignment we use the Bristol Stock Exchange (BSE) as a virtual trading environment or, 
more accurately, a minimal simulation of a limit order book financial exchange. As it is 
minimalistic it has the advantage of being both easy to understand and quick to run.

BSE acts like a dark pool continuous double auction as both buyers and sellers simultaneously bid 
towards their limit price and can see all other offers but do not know the identity of the trader. 
This has a number of implications, the main one being the difficulty of implementing the MGD trader 
in this environment as it requires non-anonymised data to calculate which trades have been updated 
each time step. This is covered in more detail in Section \ref{sec:traders_GDV}. Other traders (e.g. 
ZIP and AA) do not have this limitation.\\


\subsection{Traders} \label{sec:traders}
\subsubsection{ZIP} \label{sec:traders_ZIP}
Zero-Intelligence-Plus (ZIP) traders
\begin{itemize} \itemsep0pt
	\item What was significant --- key points
	\item Advantages/disadvantages
\end{itemize}

\subsubsection{GD-Variants} \label{sec:traders_GDV}
Rupert
\begin{itemize} \itemsep0pt
	\item Shavers / Sniper / XKCD, etc.
	\item MGD anonymised data (see BSE paragraph 2)
	\item We implemented it but couldn't do a full implementation
\end{itemize}

\pagebreak
\subsection{Adaptive Aggressive Traders} \label{sec:AA}
The adaptive aggressive algorithm relies on the principal that you can be in
two different states. One where you trade more frequently closer to your limit
price, higher chance of trading but lower profit per trade. The other where you
trade further from your limit price, which decreases the likelihood of making a
trade but increases the amount of profit you will make when you do trade.

The Adaptive-Aggressive algorithm maintains an aggressiveness value (r), which
determines how close to your limit price you will be trading. A completely
aggressive buyer ($r = 1$) will buy at their limit price, and a completely
passive trader ($r = -1$) will buy at the minimum market price (\$1 in our case)

\subsubsection{Price estimation} \label{sec:AA_price_estimation}
Short term learning reflects how the value of our aggressive is calculated and
then subsequently updated. However, first we must define two concepts. An
intra-marginal trader is a trader that bids above the equilibrium if they are a
buyer (or asks below the equilibrium if they are a seller). Whereas an
extra-marginal trader will never trade above the equilibrium if they are a
buyer (or below the equilibrium if they are a seller).

Whether our trader is acting in an extra-marginal or intra-marginal capacity
will depend upon the limit price for our order. If we have a limit price above
the equilibrium (below the equilibrium for a seller) then we are considered an
intra-marginal trader otherwise we are considered extra-marginal.

The price that we then submit will depend upon which capacity we are acting
under based on the follow equations.

\textbf{Intra-marginal buyer}
\begin{equation}
\tau =
\begin{cases}
      \hat{p}^*(1- \frac{e^{-r\theta}-1}{e^{\theta}-1}), &  \text{if r } \in (-1,0)  \\
      \hat{p}^* + (l_i-\hat{p}^*)(\frac{e^{r\theta}-1}{e^\theta-1}), & \text{if
      r } \in (0,1)
\end{cases}
\label{intrabuyer}
\end{equation}

where $\theta$ (\ref{sec:AA_long_term_learning}) measures the volatility of the
market, $l_i$ is the limit price for the buyer, $\hat{p}^*$ is the estimator of the
equilibrium price.

\textbf{Extra-marginal buyer}
\begin{equation}
\tau =
\begin{cases}
      l_i(1-\frac{e^{-r\theta}-1}{e^\theta-1} &  \text{if r } \in (-1,0)  \\
      l_i & \text{if r } \in (0,1)
\end{cases}
\label{extrabuyer}
\end{equation}

\textbf{Intra-marginal seller}
\begin{equation}
\label{intraseller}
\tau =
\begin{cases}
      \hat{p}^* + (MAX-\hat{p}^*)( \frac{e^{-r\theta}-1}{e^{\theta}-1}), &  \text{if r } \in (-1,0)  \\
      c_j + (\hat{p}^*-c_j)(1-\frac{e^{r\theta}-1}{e^\theta-1}), & \text{if r } \in (0,1)
\end{cases}
\end{equation}

where MAX is the maximum value that can be submitted to the market and $c_j$ is
the limit price for the seller.

\textbf{Extra-marginal seller}
\begin{equation}
\tau =
\begin{cases}
      c_j + (MAX-c_j)(1-\frac{e^{-r\theta}-1}{e^\theta-1} &  \text{if r } \in (-1,0)  \\
      c_j & \text{if r } \in (0,1)
\end{cases}
\label{extraseller}
\end{equation}

In the paper, there was a different value used instead of $\theta$, designed to
ensure that the curve provided by the equations was continuous. However, we
found that just using $\theta$ was still sufficiently accurate and greatly
simplified the above equations.

One of the equations is then used to determine the ideal price that we want to
be trading at. We then use the following rules to determine the actual shout
that we will make when asked to submit a new order. The $o_{bid}$ and $o_{ask}$
represent the best bid (highest) and best ask (lowest) on the market that have
not yet been accepted. The value of $\eta$ is set as a parameter and reflects
how quickly we converge on our estimated price (higher $\eta$ converge slower).

\textbf{Bidding rules for Buyer}
if ($l_i \leq o_bid$) - Submit no bid (market transacting above our limit)\\
else submit bid given by Equation \ref{bidieqn}

\begin{equation}
bid_i = \frac{o_{bid} + (\tau - o_{bid}}{\eta}
\label{bidieqn}
\end{equation}

\textbf{Bidding rules for Seller}
if ($c_j \geq o_{ask}$) - Submit no ask (market transacting below our limit)\\
else submit bid given by Equation \ref{askieqn}

\begin{equation}
ask_i = o_{ask} - \frac{o_{ask}-\tau}{\eta}
\label{askieqn}
\end{equation}

Since these all require an existing bid on the market, there were difficult
rules that were used to determine what shout to make if we were the first
trader selected to make a bid. These can be found in the Adaptive Aggressive
paper \textbf{REFERENCE GOES HERE}  but are not defined here.

Initially when we ran our algorithm we found that it was occasionally making a
loss upon receiving a new order from the scheduler with a different limit
price. The reason for this was that our previous bids had been submitted using
the limit price from the previous order. Since on the Bristol Stock Exchange it
is not possible to remove a bid, we added additional logic to notify us of the
new order. If it was found that the new order contained a limit that was
outside of our trading range, we cleared our previous bid using a stub.

\begin{itemize} \itemsep0pt
	\item Graph of all trades with projected price equilibrium
	\item \textbf{MEGA GRAPH}
\end{itemize}


\subsubsection{Short term learning} \label{sec:AA_short_term_learning}
Once we decided what capacity we are acting in we then use the following
equations to calculate our aggressiveness

$r(t+1) = r(t) + \beta_1(\delta(t) - r(t))\\\delta(t) = (1 \pm
\lambda_r)r_{shout} \pm \lambda_a$
\begin{itemize} \itemsep0pt
	\item Graphs --- \tt r \rm vs price equilibrium
\end{itemize}

\subsubsection{Long term learning} \label{sec:AA_long_term_learning}
Max
\begin{itemize} \itemsep0pt
	\item $\theta$
\end{itemize}



\section{Calibration} \label{sec:calibration}
Group hug
\begin{itemize} \itemsep0pt
	\item $\beta_1$, $\beta_2$, $\gamma$, $\eta$
	\item potential to compare statistically?
\end{itemize}


\section{Results} \label{sec:results}
Group hug
\begin{itemize} \itemsep0pt
	\item Graph: Average balance over time
	\item Statistical \textbf{anal}ysis --- why you used a certain test\\
Ed's report: ``According to the conducted Wilcoxon-Mann-Whitney two-tailed rank-sum tests, the difference in the observed efficiencies is significant ($U = 2, N_1 = N_2 = 10, p < 0.0003$)."
	\item Experiment with changing scheduler
	\item Other graphs
\end{itemize}

\section{Conclusion} \label{sec:conclusion}
Group hug
\begin{itemize} \itemsep0pt
	\item Thank you and good night
	\item Hold for applause
\end{itemize}


%\pagebreak
%\bibliographystyle{unsrt}
%\bibliography{References}
\end{document}
