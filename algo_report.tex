\documentclass{acm_proc_article-sp}

\usepackage{graphicx}
\usepackage{amsmath}
\usepackage{float}
\usepackage{natbib}
\usepackage{url}
\usepackage{amssymb}

\begin{document}
\title{Analysis of an Adaption of the Adaptive Aggressive Algorithm: Asking are AA Algorithms Actually All that Accurate in Applicable Auctions Anyway?}
\numberofauthors{3} 
\author{
  \alignauthor
    Rupert Bedford\\
    \email{rb9281@bris.ac.uk}
  \alignauthor
    Max Robinson\\
    \email{mr9388@bris.ac.uk}
  \alignauthor
    Jonathan Simmonds
    \email{js9721@bris.ac.uk}
}
\date{7 December 2012}

\maketitle
\begin{abstract}
\begin{verbatim}
                             .::::. 
                           .::::::::. 
                           ::::::::::: 
                           ':::::::::::.. 
                            :::::::::::::::' 
                             ':::::::::::. 
                               .::::::::::::::' 
                             .:::::::::::... 
                            ::::::::::::::'' 
                .:::.       '::::::::'':::: 
              .::::::::.      ':::::'  ':::: 
             .::::':::::::.    :::::    '::::. 
           .:::::' ':::::::::. :::::      ':::. 
         .:::::'     ':::::::::.:::::       '::. 
       .::::''         '::::::::::::::       '::. 
      .::''              '::::::::::::         :::... 
   ..::::                  ':::::::::'        .:' '''' 
..''''':'                    ':::::.' 
\end{verbatim}
\end{abstract}

\pagebreak

\section{Introduction}
Rupert

\begin{itemize}
	\item What is automated trading (history)
	\item What our goal was
\end{itemize}


\section{Environment}
\subsection{BSE}
For this assignment we use the Bristol Stock Exchange (BSE) as a virtual trading environment or, 
more accurately, a minimal simulation of a limit order book financial exchange. As it is 
minimalistic it has the advantage of being both easy to understand and quick to run.

BSE acts like a dark pool continuous double auction as both buyers and sellers simultaneously bid 
towards their limit price and can see all other offers but do not know the identity of the trader. 
This has a number of implications, the main one being the difficulty of implementing the MGD trader 
in this environment as it requires non-anonymised data to calculate which trades have been updated 
each time step.

\subsection{Traders}
\subsubsection{ZIP}
Jonny
\begin{itemize} \itemsep0pt
	\item What was significant --- key points
	\item Advantages/disadvantages
\end{itemize}

\subsubsection{GD-Variants}
Rupert
\begin{itemize} \itemsep0pt
	\item Shavers / Sniper / XKCD, etc.
	\item MGD anonymised data (see BSE paragraph 2)
	\item We implemented it but couldn't do a full implementation
\end{itemize}

\pagebreak
\subsection{Adaptive Aggressive Traders}
The adaptive aggressive algorithm relies on the principal that you can be in
two different states. One where you trade more frequently closer to your limit
price, higher chance of trading but lower profit per trade. The other where you
trade further from your limit price, which decreases the liklihood fo making a
trade but increases the amount of profit you will make when you do trade.

The Adaptive-Aggressive algorithm maintains an aggressiveness value (r), which
determines how close to your limit price you will be trading. A completely
aggressive buyer (r = 1) will buy at their limit price, and a completely
passive trader (r = -1) will buy at the minimum market price (\$1 in our case)

\subsubsection{Price estimation}
Short term learning reflects how the value of our aggressive is calculated and
then subsequently updated. However, first we must define two concepts. An
intra-marginal trader is a trader that bids above the equilibrium if they are a
buyer (or asks below the equilibrium if they are a seller). Whereas an
extra-marginal trader will never trade above the equilibrium if they are a
buyer (or below the equilibrium if they are a seller).

Whether our trader is acting in an extra-marginal or intra-marginal capacity
will depend upon the limit price for our order. If we have a limit price above
the equilibrium (below the equilibrium for a seller) then we are considered an
intra-marginal trader otherwise we are considered extra-marginal.

The price that we then submit will depend upon which capacity we are acting
under based on the follow equations.

\textbf{Intra-marginal buyer}

$\tau =
\begin{cases}
      \hat{p}^*(1- \frac{e^{-r\theta}-1}{e^{\theta}-1}), &  \text{if r } \in (-1,0)  \\
      \hat{p}^* + (l_i-\hat{p}^*)(\frac{e^{r\theta}-1}{e^\theta-1}), & \text{if
      r } \in (0,1)
\end{cases}$

where $\theta$ (REFERENCE) measures the volatility of the
market, $l_i$ is the limit price, $\hat{p}^*$ is the estimator of the
equilibrium price.

\textbf{Extra-marginal buyer}

$\tau =
\begin{cases}
      l_i(1-\frac{e^{-r\theta}-1}{e^\theta-1} &  \text{if r } \in (-1,0)  \\
      l_i & \text{if
      r } \in (0,1)
\end{cases}$


\begin{itemize} \itemsep0pt
	\item Graph of all trades with projected price equilibrium
	\item \textbf{MEGA GRAPH}
\end{itemize}


\subsubsection{Short term learning}
Once we decided what capacity we are acting in we then use the following
equations to calculate our aggressiveness

$r(t+1) = r(t) + \beta_1(\delta(t) - r(t))\\\delta(t) = (1 \pm
\lambda_r)r_{shout} \pm \lambda_a$
\begin{itemize} \itemsep0pt
	\item Graphs --- \tt r \rm vs price equilibrium
\end{itemize}

\subsubsection{Long term learning}
Max
\begin{itemize} \itemsep0pt
	\item $\theta$
\end{itemize}



\section{Calibration}
Group hug
\begin{itemize} \itemsep0pt
	\item $\beta_1$, $\beta_2$, $\gamma$, $\eta$
	\item potential to compare statistically?
\end{itemize}


\section{Results}
Group hug
\begin{itemize} \itemsep0pt
	\item Graph: Average balance over time
	\item Statistical \textbf{anal}ysis --- why you used a certain test\\
Ed's report: ``According to the conducted Wilcoxon-Mann-Whitney two-tailed rank-sum tests, the difference in the observed efficiencies is significant ($U = 2, N_1 = N_2 = 10, p < 0.0003$)."
	\item Experiment with changing scheduler
	\item Other graphs
\end{itemize}

\section{Conclusion}
Group hug
\begin{itemize} \itemsep0pt
	\item Thank you and good night
	\item Hold for applause
\end{itemize}


%\pagebreak
%\bibliographystyle{unsrt}
%\bibliography{References}
\end{document}
